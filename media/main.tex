%------------------------------------------------
% Packages
%------------------------------------------------
\documentclass[12pt,a4paper]{article} % Corrected!

\usepackage[a4paper, left=25mm, right=25mm, top=25mm, bottom=25mm]{geometry} % Good margins
\usepackage{tikz}
\usetikzlibrary{trees}
\usepackage[utf8]{inputenc}
\usepackage[T1]{fontenc}
\usepackage{amsmath,amssymb,amsthm}
\usepackage{eurosym}            
\usepackage[most]{tcolorbox}
\usepackage{array}
\usepackage{tabularx}
\usepackage{graphicx}
\usepackage{float}
\usepackage{tikz}
\usepackage{fancyhdr}
\usepackage{pdfpages}
\usepackage{array}

% Example/Question Box settings
\newtcolorbox{examplebox}{
  colback=gray!10,
  colframe=gray!60,
  boxrule=0.4pt,
  arc=2pt,
  left=4pt,
  right=4pt,
  top=4pt,
  bottom=4pt
}
%--------------------------------------------------
% Other Packages 
%--------------------------------------------------

%---------------------------------------------------
% Configure the header
%---------------------------------------------------
\pagestyle{fancy}
\fancyhf{} % Clear default header and footer
\fancyhead[L]{Probability Questions} % Left side of the header
\fancyhead[C]{\today}            % Center of the header (current date)
\fancyhead[R]{Page \thepage}     % Right side of the header
% Optional footer configuration
\fancyfoot[L]{Author: Morgan McKnight}
\fancyfoot[C]{}
\fancyfoot[R]{\thepage}
\title{Revision Descriptive Stats }
\author{Morgan McKnight}
\date{2025 LC}

%----------------------------------------------------------------
% BEGIN
%----------------------------------------------------------------
\begin{document}
\maketitle

%------------------------------------------------
% Mean and Mode Questions & Answers Section
%------------------------------------------------
\section{Mean and Mode Calculations}

\begin{examplebox}
\textbf{Question 1} \\
From the following set of numbers, calculate the mean \& the mode: \\
3, 12, 17, 24, 24, 31, 34, 39
\end{examplebox}

\begin{examplebox}
\textbf{Question 2} \\
From the following set of numbers, calculate the mean \& the mode: \\
6, 9, 12, 15, 15, 18, 19, 20, 23, 27, 27, 27, 30, 32
\end{examplebox}

\begin{examplebox}
\textbf{Question 3 (Higher Level Only)} \\
The mean number of points scored by a Gaelic football team per game in a 12-game season is 21. \\
If the number of points scored in 11 of the games is: \\
7, 8, 11, 15, 20, 21, 24, 26, 29, 35, 40 \\
Find the number of points scored in the 12\textsuperscript{th} game.
\end{examplebox}

\newpage
%------------------------------------------------
% Mean and Mode Detailed Answers (Full Text)
%------------------------------------------------
\section*{Detailed Answers}

\textbf{Question 1} \\
From the following set of numbers, calculate the mean \& the mode: \\
3, 12, 17, 24, 24, 31, 34, 39

\textbf{Answer} \\
Mean: 23 \\
Mode: 24

\textbf{Hint:} \\
To work out the mean we use the formula 
\[
\text{Mean} = \frac{\text{Sum of all Numbers}}{\text{Number of Numbers}}
\]
In simpler language, we have to add up all of our terms and divide by the amount of terms that we have.

To work out the mode, we simply identify the number that appears most often in the set.

If you didn't get this answer, watch the previous video to get a run-through of the method you need to use.

\vspace{1cm}

\textbf{Question 2} \\
From the following set of numbers, calculate the mean \& the mode: \\
6, 9, 12, 15, 15, 18, 19, 20, 23, 27, 27, 27, 30, 32

\textbf{Answer} \\
Mean: 20 \\
Mode: 27

\textbf{Hint:} \\
To work out the mean we use the formula 
\[
\text{Mean} = \frac{\text{Sum of all Numbers}}{\text{Number of Numbers}}
\]
In simpler language, we have to add up all of our terms and divide by the amount of terms that we have.

To work out the mode, we simply identify the number that appears most often in the set.

If you didn't get this answer, watch the previous video to get a run-through of the method you need to use.

\vspace{1cm}

\textbf{Question 3 (Higher Level Only)} \\
The mean number of points scored by a Gaelic football team per game in a 12-game season is 21. If the number of points scored in 11 of the games is: \\
7, 8, 11, 15, 20, 21, 24, 26, 29, 35, 40 \\
find the number of points scored in the 12\textsuperscript{th} game.

\textbf{Answer:} 16

\textbf{Explanation:} \\
We can rearrange the formula from Questions 1 and 2 to get: \\
\[
\text{Sum} = \text{Mean} \times \text{Number of Numbers}
\]
We know what our mean is, and we have 12 games, so there will be 12 numbers.

This means that:
\[
\text{Sum} = 21 \times 12 = 252
\]

When we add up all the points scored by the team throughout the season we should get 252.

We can add the numbers we have together, and add \( x \) for the value that we are trying to find. We know that they should equal 252, so with this equation, we can solve for \( x \), the number of points scored in the 12\textsuperscript{th} game.

\[
7 + 8 + 11 + 15 + 20 + 21 + 24 + 26 + 29 + 35 + 40 + x = 236 + x
\]

This means that:
\[
236 + x = 252
\]

To solve for \( x \), we can subtract 236 from both sides to give us:
\[
x = 16
\]
%------------------------------------------------
% Extra Mean and Mode Questions & Answers Section
%------------------------------------------------
\section{Additional Mean and Mode Practice}

\begin{examplebox}
\textbf{Question 4} \\
Find the mean and mode of the following data set: \\
5, 6, 7, 9, 10, 10, 12, 14, 15
\end{examplebox}

\begin{examplebox}
\textbf{Question 5} \\
Find the mean and mode of the following numbers: \\
18, 21, 23, 23, 25, 26, 28, 30, 32, 35
\end{examplebox}

\begin{examplebox}
\textbf{Question 6 (Higher Level Only)} \\
The mean score in a class test of 15 students was 68. If the scores of 14 students were: \\
45, 52, 55, 61, 64, 65, 70, 71, 74, 77, 80, 84, 86, 90 \\
What was the score of the 15\textsuperscript{th} student?
\end{examplebox}

\newpage

%------------------------------------------------
% Detailed Answers for Additional Practice
%------------------------------------------------
\section*{Detailed Answers (Additional Questions)}

\textbf{Question 4} \\
Find the mean and mode of the following data set: \\
5, 6, 7, 9, 10, 10, 12, 14, 15

\textbf{Answer} \\
Mean: 9.78 (rounded to 2 decimal places) \\
Mode: 10

\textbf{Explanation:} \\
Add all the numbers:
\[
5 + 6 + 7 + 9 + 10 + 10 + 12 + 14 + 15 = 88
\]
There are 9 numbers, so:
\[
\text{Mean} = \frac{88}{9} \approx 9.78
\]
The mode is 10 because it appears more than once.

\vspace{1cm}

\textbf{Question 5} \\
Find the mean and mode of the following numbers: \\
18, 21, 23, 23, 25, 26, 28, 30, 32, 35

\textbf{Answer} \\
Mean: 26.1 \\
Mode: 23

\textbf{Explanation:} \\
Add all the numbers:
\[
18 + 21 + 23 + 23 + 25 + 26 + 28 + 30 + 32 + 35 = 261
\]
There are 10 numbers, so:
\[
\text{Mean} = \frac{261}{10} = 26.1
\]
The mode is 23 because it appears twice, more than any other number.

\vspace{1cm}

\textbf{Question 6 (Higher Level Only)} \\
The mean score in a class test of 15 students was 68. \\
The scores of 14 students were: \\
45, 52, 55, 61, 64, 65, 70, 71, 74, 77, 80, 84, 86, 90

\textbf{Answer:} 82

\textbf{Explanation:} \\
We use the formula:
\[
\text{Total Sum} = \text{Mean} \times \text{Number of Scores}
\]
So the total of all 15 students' scores is:
\[
68 \times 15 = 1020
\]
Add the scores of the 14 known students:
\[
45 + 52 + 55 + 61 + 64 + 65 + 70 + 71 + 74 + 77 + 80 + 84 + 86 + 90 = 938
\]
Now subtract from the total to find the missing student's score:
\[
1020 - 938 = 82
\]
So, the 15\textsuperscript{th} student scored 82.
%------------------------------------------------
% Grouped Data: Questions & Answers Section
%------------------------------------------------
\section{Grouped Data: Mean and Median Estimates}

\begin{examplebox}
\textbf{Question 1} \\
The test results of students who recently sat a maths exam are shown below.

From the table, estimate the mean score of students who took the exam.

\begin{center}
\begin{tabular}{|c|c|}
\hline
\textbf{Score} & \textbf{No. of People} \\
\hline
0\%--20\% & 3 \\
20\%--40\% & 6 \\
40\%--60\% & 12 \\
60\%--80\% & 14 \\
80\%--100\% & 5 \\
\hline
\end{tabular}
\end{center}
\end{examplebox}

\begin{examplebox}
\textbf{Question 2} \\
From the same table, in which interval does the median lie?

\begin{center}
\begin{tabular}{|c|c|}
\hline
\textbf{Score} & \textbf{No. of People} \\
\hline
0\%--20\% & 3 \\
20\%--40\% & 6 \\
40\%--60\% & 12 \\
60\%--80\% & 14 \\
80\%--100\% & 5 \\
\hline
\end{tabular}
\end{center}
\end{examplebox}

\begin{examplebox}
\textbf{Question 3} \\
The flight delay time, in minutes, for a particular airline is shown in the table below.

\begin{enumerate}[label=(\roman*)]
\item Using mid-interval values from the table, estimate the mean flight delay time.
\item In which interval does the median lie?
\end{enumerate}

\begin{center}
\begin{tabular}{|c|c|}
\hline
\textbf{Delay Time (minutes)} & \textbf{No. of Flights} \\
\hline
0--10 & 24 \\
10--20 & 18 \\
20--30 & 11 \\
30--40 & 13 \\
40--50 & 9 \\
50--60 & 5 \\
\hline
\end{tabular}
\end{center}
\end{examplebox}

\newpage

%------------------------------------------------
% Detailed Answers: Grouped Data Questions
%------------------------------------------------
\section*{Detailed Answers (Grouped Data)}

\textbf{Question 1}

\textbf{Answer:} 56\%

\textbf{Hint:} \\
Find the mid-interval values of all the intervals. For example, the mid-interval value of the 20\%--40\% interval is 30\%.

Multiply each mid-interval value by the frequency with which it occurs.

Add each of these answers together, and then divide by the amount of students that sat the exam. This is our final answer.

If you didn't get this answer, watch the previous video to get a run-through of the method you need to use.

\vspace{1cm}

\textbf{Question 2}

\textbf{Answer:} 40\%--60\%

\textbf{Hint:} \\
The median is the number in the middle of the data set, when it is put in ascending order from smallest to biggest.

We have 40 students, so that means the median score will be the 20\textsuperscript{th}--21\textsuperscript{st} scores.

We need to find the interval in which these scores lie:

\begin{itemize}
  \item The first 3 students are in the 0\%--20\% interval,
  \item The next 6 students are in the 20\%--40\% interval (so first 9 students covered),
  \item The next 12 students are in the 40\%--60\% interval.
\end{itemize}

This means the 20\textsuperscript{th} and 21\textsuperscript{st} scores lie somewhere in the third interval: \textbf{40\%--60\%}.

If you didn't get this answer, watch the previous video to get a run-through of the method you need to use.

\vspace{1cm}

\textbf{Question 3}

\textbf{Answer:}
\begin{itemize}
  \item[(i)] 22.5 minutes
  \item[(ii)] 10--20 minutes
\end{itemize}

\textbf{Explanation:}

Find the mid-interval values of all the intervals. For example, the mid-interval value of the 0--10 interval is 5.

Multiply each mid-interval value by the frequency with which it occurs. Add all of these answers together, and then divide by the number of students that sat the exam:
\[
\frac{(5 \times 24)+(15 \times 18)+(25 \times 11)+(35 \times 13)+(45 \times 9)+(55 \times 5)}{24 + 18 + 11 + 13 + 9 + 5}
\]
\[
= \frac{120 + 270 + 275 + 455 + 405 + 275}{80} = \bm{22.5 \text{ minutes}}
\]

The median is the number in the middle of the data set when it is put in ascending order from smallest to biggest.

We have 80 students, so the median flight delay time will be the 40\textsuperscript{th}--41\textsuperscript{st} delay times.

We need to find the interval in which these delay times lie:

\begin{itemize}
  \item The first 24 delays are in the 0--10 interval,
  \item The next 18 delays are in the 10--20 interval.
\end{itemize}

This means the 40\textsuperscript{th} and 41\textsuperscript{st} delays lie somewhere in the second interval: \textbf{10--20 minutes}.
%------------------------------------------------
% Range and Interquartile Range Section
%------------------------------------------------
\section{Range and Interquartile Range}

\begin{examplebox}
\textbf{Question 1} \\
Find the range of: \\
8, 16, 3, 18, 7, 14, 6
\end{examplebox}

\begin{examplebox}
\textbf{Question 2} \\
Find the interquartile range of: \\
8, 16, 3, 18, 7, 14, 6
\end{examplebox}

\begin{examplebox}
\textbf{Question 3} \\
Find the range and interquartile range of: \\
9, 20, 28, 6, 1, 11, 24, 4, 26, 19
\end{examplebox}

\newpage

%------------------------------------------------
% Detailed Answers: Range and IQR
%------------------------------------------------
\section*{Detailed Answers (Range and Interquartile Range)}

\textbf{Question 1}

\textbf{Answer:} 15

\textbf{Hint:} \\
First, we put the numbers in ascending order from smallest to biggest.

We find the range by taking the smallest number away from the biggest number.

If you didn't get this answer, watch the previous video to get a run-through of the method you need to use.

\vspace{1cm}

\textbf{Question 2}

\textbf{Answer:} \(16 - 6 = 10\)

\textbf{Hint:} \\
First, we put the numbers in ascending order from smallest to biggest.

The interquartile range is the difference between the lower quartile and the upper quartile.

We use our formulas to find the lower quartile and the upper quartile. The formula for the lower quartile is
\[
\frac{1}{4}(n+1), \quad \text{and for the upper quartile: } \frac{3}{4}(n+1)
\]
where \(n\) is the number of terms. The answer we get from both of these formulas tells us the order of the lower/upper quartile in the sequence. For example, if we get 2 from the lower quartile formula, that means that the second term is the lower quartile.

Once you get the lower and upper quartiles, subtract these to get the interquartile range.

If you didn't get this answer, watch the previous video to get a run-through of the method you need to use.

\vspace{1cm}

\textbf{Question 3}

\textbf{Answer:} \\
Range: 27 \\
Interquartile Range: 20

\textbf{Explanation:} \\
First, we put the numbers in ascending order from smallest to biggest: \\
1, 4, 6, 9, 11, 19, 20, 24, 26, 28

We find the range by taking the smallest number away from the biggest number: \\
\[
28 - 1 = 27
\]

We use our formulas to find the lower quartile and the upper quartile. The formula for the lower quartile is:
\[
\frac{1}{4}(n+1), \quad \text{and the upper quartile: } \frac{3}{4}(n+1)
\]
where \(n = 10\). The answer we get from both of these formulas tells us the order of the lower/upper quartile in the sequence.

If we get a decimal, we take the two terms on either side of the decimal, add them together and divide by 2. For example, if we get 2.75, we take the 2\textsuperscript{nd} and 3\textsuperscript{rd} terms, add them together and divide by 2.

\textbf{Lower Quartile:}
\[
\frac{1}{4}(10+1) = \frac{11}{4} = 2.75
\]
Take the 2\textsuperscript{nd} and 3\textsuperscript{rd} terms:
\[
\frac{4 + 6}{2} = 5
\]

\textbf{Upper Quartile:}
\[
\frac{3}{4}(10+1) = \frac{33}{4} = 8.25
\]
Take the 8\textsuperscript{th} and 9\textsuperscript{th} terms:
\[
\frac{24 + 26}{2} = 25
\]

Now subtract the lower quartile from the upper quartile to get the interquartile range:
\[
25 - 5 = 20
\]
%------------------------------------------------
% Standard Deviation: Questions & Answers
%------------------------------------------------
\section{Standard Deviation}

\begin{examplebox}
\textbf{Question 1} \\
Find the standard deviation of the set: \\
\{10, 14, 16, 21, 23\} \\
Give your answer correct to 3 decimal places.
\end{examplebox}

\begin{examplebox}
\textbf{Question 2} \\
Find the standard deviation of the set: \\
\{8, 12, 19, 27, 33, 34, 36, 36\} \\
Give your answer correct to 1 decimal place.
\end{examplebox}

\begin{examplebox}
\textbf{Question 3} \\
Find the standard deviation of the numbers: \\
\{6, 8, 11, 14, 15, 16, 18, 24, 25, 25, 26, 28, 38, 40\} \\
Give your answer correct to 1 decimal place.
\end{examplebox}

\newpage

%------------------------------------------------
% Detailed Answers: Standard Deviation
%------------------------------------------------
\section*{Detailed Answers (Standard Deviation)}

\textbf{Question 1}

\textbf{Answer:} 4.707

\textbf{Hint:} \\
To find the standard deviation we use the formula:
\[
\sigma = \sqrt{\frac{1}{n} \sum(x - \mu)^2}
\]
where each \(x\) is an observation, \(\mu\) is the mean, and \(n\) is the number of values in the data set.

Find the mean by adding up all of the numbers and dividing by the amount of values.  
Here, the mean is equal to:
\[
\mu = \frac{10 + 14 + 16 + 21 + 23}{5} = 16.8
\]

\(\sum\) means to evaluate \((x - \mu)^2\) for each observation and add up the results.

If you didn't get this answer, watch this tutorial to get a run-through of the method you need to use.

\vspace{1cm}

\textbf{Question 2}

\textbf{Answer:} 10.5

\textbf{Hint:} \\
To find the standard deviation we use the formula:
\[
\sigma = \sqrt{\frac{1}{n} \sum(x - \mu)^2}
\]
where each \(x\) is an observation, \(\mu\) is the mean, and \(n\) is the number of values in the data set.

Find the mean by adding up all of the numbers and divide by the number of values.  
Here, the mean is equal to:
\[
\mu = \frac{8 + 12 + 19 + 27 + 33 + 34 + 36 + 36}{8} = 25.625
\]

\(\sum\) means to evaluate \((x - \mu)^2\) for each observation and add up the results.

If you didn't get this answer, watch this tutorial to get a run-through of the method you need to use.

\vspace{1cm}

\textbf{Question 3}

\textbf{Answer:} 9.9

\textbf{Explanation:} \\
To find the standard deviation we use the formula:
\[
\sigma = \sqrt{\frac{1}{n} \sum(x - \mu)^2}
\]
where each \(x\) is an observation, \(\mu\) is the mean, and \(n\) is the number of values in the data set.

Find the mean by adding up all of the numbers and dividing by the amount of numbers:
\[
\mu = \frac{6 + 8 + 11 + 14 + 15 + 16 + 18 + 24 + 25 + 25 + 26 + 28 + 38 + 40}{14} = 21
\]

Now evaluate \((x - \mu)^2\) for each \(x\), then sum those results:
\[
\sum(x - \mu)^2 = 1378
\]

Divide this sum by 14, then take the square root:
\[
\sigma = \sqrt{\frac{1378}{14}} = \sqrt{98.43} = 9.9
\]

If you didn't get this answer, watch this tutorial to get a run-through of the method you need to use.
%------------------------------------------------
% Calculator-Based Standard Deviation Section
%------------------------------------------------
\section{Standard Deviation Using a Calculator}

\begin{examplebox}
\textbf{Question 1} \\
Using a calculator, calculate the standard deviation of the set: \\
\( A = \left\lbrace 9, 13, 14, 15, 19 \right\rbrace \)
\end{examplebox}

\begin{examplebox}
\textbf{Question 2} \\
Using your calculator, calculate the standard deviation of the set: \\
\( B = \left\lbrace 2, 5, 6, 7, 3, 8, 11 \right\rbrace \)
\end{examplebox}

\begin{examplebox}
\textbf{Question 3} \\
Using a calculator, calculate the standard deviation of the set: \\
\( C = \left\lbrace 3.2, 4.6, 2.8, 5.2, 4.4 \right\rbrace \)
\end{examplebox}

\newpage

%------------------------------------------------
% Detailed Answers: Calculator Standard Deviation
%------------------------------------------------
\section*{Detailed Answers (Calculator-Based)}

\textbf{Question 1}

\textbf{Answer:} \( \sigma_{A} = 3.22 \)

\textbf{Explanation:} \\
Input the numbers in the set into your calculator using the statistics function.  

Use the functions on the calculator as explained in the video to work out the standard deviation \( \sigma_{x} \).

If you didn't get this answer, watch the previous video to get a run-through of the method you need to use.

\vspace{1cm}

\textbf{Question 2}

\textbf{Answer:} \( \sigma_{B} = 2.83 \)

\textbf{Explanation:} \\
Input the numbers in the set into your calculator using the statistics function.  

Use the functions on the calculator as explained in the video to work out the standard deviation \( \sigma_{x} \).

If you didn't get this answer, watch the previous tutorial for a run-through of the method you need to use.

\vspace{1cm}

\textbf{Question 3}

\textbf{Answer:} \( \sigma_{C} = 0.898 \)

\textbf{Explanation:} \\
Input the numbers in the set into your calculator using the statistics function.  

Use the functions on the calculator as explained in the video to work out the standard deviation \( \sigma_{x} \).
%------------------------------------------------
% Stem and Leaf Plot Section
%------------------------------------------------
\section{Stem and Leaf Plots}

\begin{examplebox}
\textbf{Question 1} \\
The ages of 20 random people in a shopping centre at a given time were recorded. The ranked results are as follows: \\
17, 16, 20, 28, 29, 24, 30, 32, 32, 32, \\
36, 38, 39, 43, 47, 51, 57, 57, 58, 72

\begin{itemize}
  \item[(a)] Display these results in a stem and leaf plot.
  \item[(b)] Calculate the range of the ages.
\end{itemize}
\end{examplebox}

\begin{examplebox}
\textbf{Question 2} \\
The time (in seconds) taken for a group of 3rd years and a group of 6th years to complete a 100m sprint are as follows:

\begin{itemize}
  \item 3rd Years: 13.8, 14.2, 14.4, 14.6, 14.7, 14.9, 15, 15.1, 15.6, 15.8, 17.5, 17.9
  \item 6th Years: 13.2, 13.6, 13.9, 14.5, 14.5, 14.7, 15.8, 15.8, 16.2, 16.3, 17, 17.1
\end{itemize}

Display this information on a back-to-back stem and leaf plot.
\end{examplebox}
\begin{examplebox}
\textbf{Question 3} \\
Calculate the interquartile range for the 3rd years' time and the 6th years' time in Question 2 above.
\end{examplebox}

\newpage

%------------------------------------------------
% Detailed Answers: Stem and Leaf Plot
%------------------------------------------------
\section*{Detailed Answers (Stem and Leaf Plots)}

\textbf{Question 1}

\textbf{(a) Stem and Leaf Plot:}

\begin{center}
\begin{tabular}{r|l}
\textbf{Stem} & \textbf{Leaf} \\
\hline
1 & 6 7 \\
2 & 0 4 8 9 \\
3 & 0 2 2 2 \\
3 & 6 8 9 \\
4 & 3 7 \\
5 & 1 7 7 8 \\
7 & 2 \\
\end{tabular}
\end{center}

\textbf{Key:} 3\,|\,6 means 36

\vspace{0.5cm}

\textbf{(b) Range:} \(72 - 16 = 56\)

\textbf{Explanation:} \\
Don't forget a key for your plot – this can be any value from the data set.

The range equals the largest minus the smallest value.

If you didn't get this answer, watch the previous video to get a run-through of the method you need to use.

\vspace{1cm}

\textbf{Question 2}

\textbf{Back-to-Back Stem and Leaf Plot:}

\begin{center}
\begin{tabular}{r | l | l}
\textbf{6th Years} & \textbf{Stem} & \textbf{3rd Years} \\
\hline
2 & 13 & 8 \\
6 3 & 14 & 2 4 6 7 9 \\
8 8 & 15 & 0 1 6 8 \\
3 2 & 16 &  \\
1 0 & 17 & 5 9 \\
\end{tabular}
\end{center}

\textbf{Keys:} \\
6\,|\,13 = 13.6 (6th Year) \\
13\,|\,8 = 13.8 (3rd Year)

\textbf{Explanation:} \\
Make sure to put your values in order for each row in the stem and leaf.

We list the values backwards on the left-hand side and forwards on the right-hand side.

For the values like 15.0 and 17.0 be sure to put a 0 on the leaf side.

Always include a key for each side.
\vspace{1cm}

\textbf{Question 3}

\textbf{Answer:} \\
3rd Year IQR: \(15.7 - 14.5 = 1.2\) \\
6th Year IQR: \(16.25 - 14.2 = 2.05\)

\textbf{Explanation:}

First we will work on the 3rd year data and find the median:

\[
\cancel{13.8},\ \cancel{14.2},\ \cancel{14.4},\ \cancel{14.6},\ \cancel{14.7},\ 14.9,\ 15,\ \cancel{15.1},\ \cancel{15.6},\ \cancel{15.8},\ \cancel{17.5},\ \cancel{17.9}
\]
\[
\text{Median} = \frac{14.9 + 15}{2} = 14.95
\]

Next, we find the lower quartile by finding the median of the data below the median:

\[
13.8,\ 14.2,\ 14.4,\ 14.6,\ 14.7,\ 14.9,\ \mathbf{14.95},\ 15,\ 15.1,\ 15.6,\ 15.8,\ 17.5,\ 17.9
\]
\[
\cancel{13.8},\ \cancel{14.2},\ 14.4,\ 14.6,\ \cancel{14.7},\ \cancel{14.9},\ \mathbf{14.95},\ 15,\ 15.1,\ 15.6,\ 15.8,\ \cancel{17.5},\ \cancel{17.9}
\]

Lower Quartile:
\[
\frac{14.4 + 14.6}{2} = 14.5
\]

Now find the upper quartile by finding the median of the data above 14.95:

\[
13.8,\ 14.2,\ 14.4,\ 14.6,\ 14.7,\ 14.9,\ \mathbf{14.95},\ \cancel{15},\ \cancel{15.1},\ 15.6,\ 15.8,\ \cancel{17.5},\ \cancel{17.9}
\]

Upper Quartile:
\[
\frac{15.6 + 15.8}{2} = 15.7
\]

So the interquartile range for 3rd years is:
\[
\text{IQR} = 15.7 - 14.5 = 1.2
\]

---

Repeat the same process for the 6th year data:

\textbf{Ordered data:} \\
13.2, 13.6, 13.9, 14.5, 14.5, 14.7, 15.8, 15.8, 16.2, 16.3, 17, 17.1

\textbf{Median:}
\[
\cancel{13.2},\ \cancel{13.6},\ \cancel{13.9},\ \cancel{14.5},\ \cancel{14.5},\ 14.7,\ 15.8,\ \cancel{15.8},\ \cancel{16.2},\ \cancel{16.3},\ \cancel{17},\ \cancel{17.1}
\]
\[
\text{Median} = \frac{14.7 + 15.8}{2} = 15.25
\]

\textbf{Lower Quartile:}
\[
\cancel{13.2},\ 13.6,\ \cancel{13.9},\ 14.5,\ 14.5,\ \cancel{14.7}
\Rightarrow \frac{13.6 + 14.5}{2} = 14.05 \approx 14.2
\]

\textbf{Upper Quartile:}
\[
\cancel{15.8},\ 15.8,\ 16.2,\ 16.3,\ 17,\ \cancel{17.1}
\Rightarrow \frac{16.2 + 16.3}{2} = 16.25
\]

So the interquartile range for 6th years is:
\[
\text{IQR} = 16.25 - 14.2 = 2.05
\]
%------------------------------------------------
% Histogram Questions and Answers (No Charts)
%------------------------------------------------
\section{Histograms and Distribution}

\begin{examplebox}
\textbf{Question 1} \\
The histogram below shows the daily screen time in hours of a sample of 75 students.

Copy and complete the following table. Give your answers correct to two decimal places where necessary.
\end{examplebox}

\begin{examplebox}
\textbf{Question 2} \\
Construct a histogram to display the following data:

\begin{center}
\begin{tabular}{|c|c|}
\hline
\textbf{Interval (hrs)} & \textbf{Frequency} \\
\hline
0--10 & 3 \\
10--20 & 7 \\
20--30 & 12 \\
30--40 & 20 \\
40--50 & 14 \\
50--60 & 9 \\
\hline
\end{tabular}
\end{center}
\end{examplebox}

\begin{examplebox}
\textbf{Question 3} \\
Using the table from Question 2, find:  
(a) The modal interval  
(b) The interval in which the median lies.
\end{examplebox}

\newpage
\section*{Detailed Answers}

\textbf{Answer to Question 1:}

\begin{center}
\begin{tabular}{@{}lll@{}}
\toprule
\textbf{Interval (hrs)} & \textbf{Frequency} & \textbf{Percentage} \\
\midrule
0--10  & 3  & 4.62\% \\
10--20 & 7  & 10.77\% \\
20--30 & 12 & 18.46\% \\
30--40 & 20 & 30.77\% \\
40--50 & 14 & 21.54\% \\
50--60 & 9  & 13.85\% \\
\midrule
\textbf{Total} & 65 & 100.00\% \\
\bottomrule
\end{tabular}
\end{center}

\textbf{Explanation:}  
We read the frequency of each interval from the histogram.  
To calculate the percentage we use the formula:
\[
\text{Percentage} = \frac{\text{Frequency}}{\text{Total}} \times 100
\]

\vspace{0.5cm}

\textbf{Answer to Question 2:}

This question requires the student to draw a histogram using the table above.

\textbf{Explanation:}  
Use a ruler to draw axes and bars. Make sure intervals on the x-axis and y-axis are evenly spaced. Label both axes.

\vspace{0.5cm}

\textbf{Answer to Question 3:}

(a) Modal interval = \textbf{30--40 hours} \\
(b) Median interval = \textbf{30--40 hours}

\textbf{Explanation:}

The mode is the most frequent class → 30--40 hours (20 people).  
The total number of people is 65. The middle value is the 33rd.

Cumulative frequencies:
\begin{itemize}
    \item 0--10: 3
    \item 10--20: 3 + 7 = 10
    \item 20--30: 10 + 12 = 22
    \item 30--40: 22 + 20 = 42
\end{itemize}

Since the 33rd person lies within the 30--40 range, it is the median interval.
%------------------------------------------------
% Distribution and Central Tendency
%------------------------------------------------
\section{Shape of Distributions and Central Tendency}

\begin{examplebox}
\textbf{Question 1} \\
If a data set is normally distributed, which of the following is true:

\begin{itemize}
  \item[A.] mean = median = mode
  \item[B.] mean < median < mode
  \item[C.] mean > median > mode
\end{itemize}
\end{examplebox}

\begin{examplebox}
\textbf{Question 2} \\
Describe the shape of the distribution shown below:
\end{examplebox}
\begin{figure}[H]
    \centering
    \includegraphics[width=0.5\linewidth]{distq2.png}
    \caption{Fig 1}
    \label{fig:enter-label}
\end{figure}
\begin{examplebox}
\textbf{Question 3 (Higher Level Only)} \\
Consider the distribution below. Labelled at A, B and C are estimated values of the distribution's central tendencies: mean, median and mode.

Match each letter to its corresponding central tendency.
\end{examplebox}
\begin{figure}[H]
    \centering
    \includegraphics[width=0.5\linewidth]{distQ3.png}
    \caption{Enter Caption}
    \label{fig:enter-label}
\end{figure}
\newpage
\section*{Detailed Answers}

\textbf{Answer to Question 1:} \\
\textbf{A. mean = median = mode}

\textbf{Explanation:} \\
Given the symmetric nature of a normal distribution, all three measures of central tendency (mean, median, and mode) are equal. 

\vspace{0.5cm}

\textbf{Answer to Question 2:} \\
\textbf{Right-tailed / Right-skewed / Positively skewed}

\textbf{Explanation:} \\
As the tail of the distribution extends in the right/positive direction, it is a right-tailed (positively skewed) distribution.  
If you didn't get this answer, watch the previous video to get a run-through of the method you need to use.

\vspace{0.5cm}

\textbf{Answer to Question 3:} \\
\begin{itemize}
  \item A is the \textbf{mode}
  \item B is the \textbf{median}
  \item C is the \textbf{mean}
\end{itemize}

\textbf{Explanation:} \\
This distribution is right-skewed (positively skewed). In such a distribution:

\[
\text{Mean} > \text{Median} > \text{Mode}
\]

That means the mean is the smallest value (C), the median is next (B), and the mode (the peak) is the largest (A).  
If you didn't get this answer, watch the previous video to get a run-through of the method you need to use.

%------------------------------------------------
% Scatterplots with Data Tables and Answers
%------------------------------------------------
\section{Scatterplots and Correlation}

\begin{examplebox}
\textbf{Question 1} \\
Construct a scatterplot to display the data in the table below. \\
Hence, describe the correlation between the rainfall and the number of customers.

\begin{center}
\begin{tabular}{|c|c|}
\hline
\textbf{Rainfall (mm)} & \textbf{Customers} \\
\hline
0  & 80 \\
2  & 78 \\
4  & 75 \\
6  & 70 \\
8  & 65 \\
10 & 60 \\
12 & 55 \\
\hline
\end{tabular}
\end{center}
\end{examplebox}

\begin{examplebox}
\textbf{Question 2} \\
Construct a scatterplot to display the data in the table below.

\begin{center}
\begin{tabular}{|c|c|}
\hline
\textbf{Temperature (°C)} & \textbf{Ice Cream Sales} \\
\hline
14 & 40 \\
16 & 45 \\
18 & 55 \\
20 & 65 \\
22 & 75 \\
24 & 85 \\
26 & 95 \\
\hline
\end{tabular}
\end{center}
\end{examplebox}

\begin{examplebox}
\textbf{Question 3} \\
The scatterplot below shows the results of students who took a maths and a physics exam. \\
Complete the table below stating the results of each student in their maths and physics exams.
\end{examplebox}
\begin{figure}[H]
    \centering
    \includegraphics[width=0.5\linewidth]{scatter3.png}
    \caption{q3}
    \label{fig:enter-label}
\end{figure}
\newpage
\section*{Detailed Answers}

\textbf{Answer to Question 1:} \\
There is a strong negative correlation between rainfall and number of customers.

\textbf{Explanation:} \\
For the scatterplot, the top row (rainfall) is the x-axis, and the second row (customers) is the y-axis.  
The plotted points move "downhill" from left to right — this indicates a strong negative correlation as the points are relatively close together.
\begin{figure}[H]
    \centering
    \includegraphics[width=0.5\linewidth]{scatter1.png}
    \caption{Enter Caption}
    \label{fig:enter-label}
\end{figure}
\vspace{0.5cm}

\textbf{Answer to Question 2:} \\
This question requires plotting the data from the table above.

\textbf{Explanation:} \\
Use temperature as the x-axis and ice cream sales as the y-axis.  
Plot each point accurately on a grid.  
The pattern should suggest a strong positive correlation.
\begin{figure}[H]
    \centering
    \includegraphics[width=0.5\linewidth]{scatter2.png}
    \caption{q2}
    \label{fig:enter-label}
\end{figure}
\vspace{0.5cm}

\textbf{Answer to Question 3:}

\begin{center}
\begin{tabular}{|c|c|}
\hline
\textbf{Maths} & \textbf{Physics} \\
\hline
50 & 52 \\
55 & 58 \\
60 & 60 \\
65 & 64 \\
70 & 68 \\
75 & 72 \\
80 & 78 \\
85 & 82 \\
90 & 88 \\
95 & 92 \\
\hline
\end{tabular}
\end{center}

\textbf{Explanation:} \\
Each point on the scatterplot corresponds to a student’s results in maths and physics.  
Write the matching coordinates into the table.  
The order does not matter as long as the pairs are correct.
%------------------------------------------------
% Describing Correlation in Scatterplots
%------------------------------------------------
\section{Describing Correlation from Scatterplots}

\begin{examplebox}
\textbf{Question 1} \\
The number of hours spent studying for a particular test and the students' test result is displayed on the scatterplot below.

Describe the correlation between the number of hours spent studying and test result.
\end{examplebox}
\begin{figure}[H]
    \centering
    \includegraphics[width=0.5\linewidth]{scatter4.png}
    \caption{Enter Caption}
    \label{fig:enter-label}
\end{figure}
\begin{examplebox}
\textbf{Question 2} \\
A test was conducted to see if there is a relationship between a person's height and the number of push-ups they can do. The results are displayed in the scatterplot below.

Describe the correlation between the two variables.
\end{examplebox}
\begin{figure}[H]
    \centering
    \includegraphics[width=0.75\linewidth]{scatter5.png}
    \caption{Enter Caption}
    \label{fig:enter-label}
\end{figure}
\begin{examplebox}
\textbf{Question 3} \\
A Credit Union studied the relationship between a number of its members' weekly savings versus the number of take-aways they buy per week. The results are displayed in the scatterplot below.

Describe the correlation between the two variables.
\end{examplebox}
\begin{figure}
    \centering
    \includegraphics[width=0.75\linewidth]{scatter6.png}
    \caption{Enter Caption}
    \label{fig:enter-label}
\end{figure}
\newpage
\section*{Detailed Answers}

\textbf{Answer to Question 1:} \\
There is a \textbf{strong positive correlation} between the two variables.

\textbf{Explanation:} \\
As the data is moving "uphill", it shows a positive correlation.  
Because the data points are relatively close together and trending in the same direction, the relationship is considered strong.  

\vspace{0.5cm}

\textbf{Answer to Question 2:} \\
There is a \textbf{moderate negative correlation} between the two variables.

\textbf{Explanation:} \\
As the data is moving "downhill", there is a negative correlation.  
Since the points are somewhat spread out but still trending in the same direction, the correlation is considered moderate.  

\vspace{0.5cm}

\textbf{Answer to Question 3:} \\
There is a \textbf{weak negative correlation} between the two variables.

\textbf{Explanation:} \\
As the data is moving "downhill", it indicates a negative correlation.  
However, the points are quite spread out, so the strength of the correlation is weak.  

%------------------------------------------------
% Correlation Coefficient with Tables
%------------------------------------------------
\section{Correlation Coefficient and Interpretation}

\begin{examplebox}
\textbf{Question 1} \\
The data below states the number of years’ work experience a worker has versus their salary. \\
Calculate, correct to two decimal places, the correlation coefficient of this data.  
Hence, describe the strength of the correlation.

\begin{center}
\begin{tabular}{|c|c|}
\hline
\textbf{Experience (years)} & \textbf{Salary (€1000s)} \\
\hline
1 & 25 \\
2 & 28 \\
3 & 35 \\
4 & 40 \\
5 & 45 \\
6 & 50 \\
7 & 55 \\
\hline
\end{tabular}
\end{center}
\end{examplebox}

\begin{examplebox}
\textbf{Question 2} \\
The data below represents a student's grade point average (GPA) and their screen time in hours on a particular weekend day. \\
Calculate the correlation coefficient of the data, correct to two decimal places.

\begin{center}
\begin{tabular}{|c|c|}
\hline
\textbf{GPA} & \textbf{Screen Time (hrs)} \\
\hline
3.8 & 2.0 \\
3.6 & 2.5 \\
3.5 & 3.0 \\
3.2 & 3.5 \\
3.0 & 4.0 \\
2.8 & 4.5 \\
2.6 & 5.0 \\
2.4 & 5.5 \\
\hline
\end{tabular}
\end{center}
\end{examplebox}

\begin{examplebox}
\textbf{Question 3} \\
A student wishes to determine whether a relationship between their class’s results in maths and physics exists.  
They select the grades of eight random students.  
Calculate, correct to two decimal places, the correlation coefficient of the data.  
Hence, describe the relationship between the data.

\begin{center}
\begin{tabular}{|c|c|}
\hline
\textbf{Maths} & \textbf{Physics} \\
\hline
65 & 60 \\
70 & 68 \\
75 & 70 \\
80 & 75 \\
85 & 78 \\
90 & 85 \\
95 & 88 \\
100 & 92 \\
\hline
\end{tabular}
\end{center}
\end{examplebox}

\newpage
\section*{Detailed Answers}

\textbf{Answer to Question 1:} \\
\[
r \equiv 0.85
\]

This represents a \textbf{strong, positive correlation}.

\textbf{Hint:} \\
We input the data into our calculator using the statistics/correlation function and find \( r \), the correlation coefficient.

If you didn't get this answer, watch the previous video to get a run-through of the method you need to use.

\vspace{0.5cm}

\textbf{Answer to Question 2:} \\
This question requires calculation of the correlation coefficient based on the GPA and screen time data.

\textbf{Explanation:} \\
Use GPA as the x-values and screen time as the y-values.  
Input them into your calculator in statistics mode and compute \( r \), correct to two decimal places.

\vspace{0.5cm}

\textbf{Answer to Question 3:} \\
\[
r = 0.55
\]

There is a \textbf{positive relationship} between the maths and physics results.

\textbf{Explanation:} \\
We input the maths and physics data into our calculator.  
The result \( r = 0.55 \) indicates a moderate positive correlation between the two sets of results.

\end{document}
